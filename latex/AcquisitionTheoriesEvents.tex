%
% Annual Cognitive Science Conference
% Sample LaTeX Two-Page Summary -- Proceedings Format
%

% Original : Ashwin Ram (ashwin@cc.gatech.edu)       04/01/1994
% Modified : Johanna Moore (jmoore@cs.pitt.edu)      03/17/1995
% Modified : David Noelle (noelle@ucsd.edu)          03/15/1996
% Modified : Pat Langley (langley@cs.stanford.edu)   01/26/1997
% Latex2e corrections by Ramin Charles Nakisa        01/28/1997
% Modified : Tina Eliassi-Rad (eliassi@cs.wisc.edu)  01/31/1998
% Modified : Trisha Yannuzzi (trisha@ircs.upenn.edu) 12/28/1999 (in process)
% Modified : Mary Ellen Foster (M.E.Foster@ed.ac.uk) 12/11/2000
% Modified : Ken Forbus                              01/23/2004
% Modified : Eli M. Silk (esilk@pitt.edu)            05/24/2005
% Modified : Niels Taatgen (taatgen@cmu.edu)         10/24/2006
% Modified : David Noelle (dnoelle@ucmerced.edu)     11/19/2014

%% Change "letterpaper" in the following line to "a4paper" if you must.

\documentclass[10pt,letterpaper]{article}

\usepackage{cogsci}
\usepackage{pslatex}
\usepackage{apacite}


\title{Grounding Language Acquisition in Intuitive Theories and Event Cognition}

\author{{\large \bf Eva Wittenberg (ewittenberg@ucsd.edu)} \\
  Department of Linguistics,
  University of California, San Diego \\
  9500 Gilman Dr.,
  La Jolla, CA 92093-0108 USA \\
  \AND {\large \bf Melissa Kline (mekline@mit.edu)} \\
  Department of Psychology,
  Harvard University \\
  33 Kirkland St.,
  Cambridge, MA 02138 USA \\
  \AND {\large \bf Joshua K. Hartshorne (joshua.hartshorne@bc.edu)} \\
  Department of Psychology,
  Boston College \\
  140 Commonwealth Ave,
  Chestnut Hill, MA 02467}


\begin{document}

\maketitle

\begin{quote}
\small
\textbf{Keywords:}
add your choice of indexing terms or keywords; kindly use a
semicolon; between each term
\end{quote}

\section{Introduction}

The entire contribution of a short summary submission (including
figures, references, and anything else) can be no longer than two
pages. This short summary format is to be used for workshop and
tutorial descriptions, symposia summaries, and publication-based
presentation extended abstracts. Unlike submitted research papers,
short summary submissions should \emph{not} begin with a separate
abstract. Prior to the first section of the short summary, there
should be the header ``{\bf Keywords:}'' followed by a list of
descriptive keywords separated by semicolons, all in 9~point font, as
shown above.

The text of the paper should be formatted in two columns with an
overall width of 7 inches (17.8 cm) and length of 9.25 inches (23.5
cm), with 0.25 inches between the columns. Leave two line spaces
between the last author listed and the text of the paper. The left
margin should be 0.75 inches and the top margin should be 1 inch.
\textbf{The right and bottom margins will depend on whether you use
  U.S. letter or A4 paper, so you must be sure to measure the width of
  the printed text.} Use 10~point Times Roman with 12~point vertical
spacing, unless otherwise specified.

The title should be in 14~point, bold, and centered. The title should
be formatted with initial caps (the first letter of content words
capitalized and the rest lower case). Each author's name should appear
on a separate line, 11~point bold, and centered, with the author's
email address in parentheses. Under each author's name list the
author's affiliation and postal address in ordinary 10~point type.

Indent the first line of each paragraph by 1/8~inch (except for the
first paragraph of a new section). Do not add extra vertical space
between paragraphs.

\section{Goals and Scope}

FUBAR

\section{Workshop Organization}

\noindent The workshop will be organized around a set of thirty-minute
presentations (including Q\&A). The presentations will range from
theoretical overviews to detailed discussion of specific
phenomena. Interspersed coffee breaks will help spur discussion about
promising avenues for future research and help build a common
vocabulary and agenda.

\subsection{Workshop Organizers}

\textbf{Eva Wittenberg} is FUBAR. \textbf{Melissa Kline} is
FUBAR. \textbf{Joshua K. Hartshorne} is an assistant professor of
Psychology at Boston College. His work focuses on the interaction
between conceptual and linguistic representations in both processing
and acquisition \cite{HOSULS,HOT2015}.

FUBAR

\section{Acknowledgments}

FUBAR - funding??


\section{References Instructions}

Follow the APA Publication Manual for citation format, both within the
text and in the reference list, with the following exceptions: (a) do
not cite the page numbers of any book, including chapters in edited
volumes; (b) use the same format for unpublished references as for
published ones. Alphabetize references by the surnames of the authors,
with single author entries preceding multiple author entries. Order
references by the same authors by the year of publication, with the
earliest first.

Use a first level section heading, ``{\bf References}'', as shown
below. Use a hanging indent style, with the first line of the
reference flush against the left margin and subsequent lines indented
by 1/8~inch. Below are example references for a conference paper, book
chapter, journal article, dissertation, book, technical report, and
edited volume, respectively.

\nocite{ChalnickBillman1988a}
\nocite{Feigenbaum1963a}
\nocite{Hill1983a}
\nocite{OhlssonLangley1985a}
% \nocite{Lewis1978a}
\nocite{Matlock2001}
\nocite{NewellSimon1972a}
\nocite{ShragerLangley1990a}


\bibliographystyle{apacite}

\setlength{\bibleftmargin}{.125in}
\setlength{\bibindent}{-\bibleftmargin}

\bibliography{CogSci_Template}


\end{document}

%%% Local Variables:
%%% mode: latex
%%% TeX-master: t
%%% End:
