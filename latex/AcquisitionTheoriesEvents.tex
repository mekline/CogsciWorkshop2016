%
% Annual Cognitive Science Conference
% Sample LaTeX Two-Page Summary -- Proceedings Format
%

% Original : Ashwin Ram (ashwin@cc.gatech.edu)       04/01/1994
% Modified : Johanna Moore (jmoore@cs.pitt.edu)      03/17/1995
% Modified : David Noelle (noelle@ucsd.edu)          03/15/1996
% Modified : Pat Langley (langley@cs.stanford.edu)   01/26/1997
% Latex2e corrections by Ramin Charles Nakisa        01/28/1997
% Modified : Tina Eliassi-Rad (eliassi@cs.wisc.edu)  01/31/1998
% Modified : Trisha Yannuzzi (trisha@ircs.upenn.edu) 12/28/1999 (in process)
% Modified : Mary Ellen Foster (M.E.Foster@ed.ac.uk) 12/11/2000
% Modified : Ken Forbus                              01/23/2004
% Modified : Eli M. Silk (esilk@pitt.edu)            05/24/2005
% Modified : Niels Taatgen (taatgen@cmu.edu)         10/24/2006
% Modified : David Noelle (dnoelle@ucmerced.edu)     11/19/2014

%% Change "letterpaper" in the following line to "a4paper" if you must.

\documentclass[10pt,letterpaper]{article}

\usepackage{hyperref}
\hypersetup{
    colorlinks=true,       % false: boxed links; true: colored links
    citecolor=black,        % color of links to bibliography
    urlcolor=cyan           % color of external links
    }

\usepackage{cogsci}
\usepackage{pslatex}
\usepackage{apacite}


% \usepackage[disable]{todonotes} % notes not showed
\usepackage[]{todonotes}   % notes showed

\newcommand{\mtodo}[2][]
{\todo[caption={#2}, size=\small, #1, color = green, inline]{\renewcommand{\baselinestretch}{1}\selectfont \textbf{TG}: #2}~}

\newcommand{\jtodo}[2][]
{\todo[caption={#2}, size=\small, #1, color = orange, inline]{\renewcommand{\baselinestretch}{1}\selectfont \textbf{JH}: #2}~}

\newcommand{\etodo}[2][]
{\todo[caption={#2}, size=\small, #1, color = red, inline]{\renewcommand{\baselinestretch}{1}\selectfont \textbf{LN}: #2}~}

\title{Talking about Thoughts\\ Grounding Language Acquisition in Intuitive Theories and Event Cognition}

\author{{\large \bf Eva Wittenberg (ewittenberg@ucsd.edu)} \\
  Department of Linguistics,
  University of California, San Diego \\
  9500 Gilman Dr.,
  La Jolla, CA 92093-0108 USA \\
  \AND {\large \bf Melissa Kline (mekline@mit.edu)} \\
  Department of Psychology,
  Harvard University \\
  33 Kirkland St.,
  Cambridge, MA 02138 USA \\
  \AND {\large \bf Joshua K. Hartshorne (joshua.hartshorne@bc.edu)} \\
  Department of Psychology,
  Boston College \\
  140 Commonwealth Ave,
  Chestnut Hill, MA 02467}


\begin{document}

\maketitle

\begin{quote}
\small
\textbf{Keywords:}
language; language acquisition; concepts; event cognition; cognitive
development; intuitive theories; argument structure
\end{quote}

\section{Introduction}

Language is a powerful tool for moving thoughts from the
head of one person to another. Thus, any theory of language must make
contact with theories of thought and conceptual representation. Any
theory of language acquisition must explain how children link words to
concepts, whether those concepts were pre-existing or created during
the process of language acquisition. Conversely, theories of
conceptual representation are constrained by the need to support
language and language acquisition. Indeed, there is a rich tradition
of cognitive science research that explicitly tackles these issues \cite{Pinker1989,Clark2004,Bowerman1989}.

Right now is a particularly opportune time to assess our
current understanding of how language acquisition might be grounded in
thought -- particularly focusing beyond the better-explored domains of
objects and kinds to events. Recent years have seen the emergence of a
robust psychological literature and event representations \cite{Radvansky2014}. There has
likewise been explosive growth in work on concepts within the
``intuitive theory'' or ``Theory Theory'' framework
\cite{Gopnikinpress,Battaglia2013,HOT2015}. On such accounts, concepts are
embedded in robust theories -- much like the theories used by
scientists -- and derive their meaning from their role in those
theories. Importantly, our growing understanding of concepts
is not limited to mature representations: Work on infant and child
conceptual representations has been particularly productive \cite{Gopnikinpress,Ettinger,Henrik2015}.

Language research has similarly seen rapid progress. In this workshop,
we focus in particular on verbs. Linguists now have well-specified,
articulated theories of the semantics of verbs, which is necessarily
(part of) a theory of event representation \cite{Levin2011,Levin2005}. Importantly, there is now increasingly strong evidence for their psychological reality and role in language acquisition
\cite{Ambridge2013,HOSULS}. While these theories bear certain
similarities to the infant cognition work (e.g., representations of
agency, intentionality, and causation play a key role), in other ways
they diverge (e.g., they have no clear analog to an intuitive theory).

This workshop brings language acquisition researchers
together with experts in event cognition, intuitive theories, and
cognitive development in order to try to understand how the
developments in these disparate-yet-linked fields inform one
another. Because language must make contact with thought -- otherwise,
how do we communicate -- it is likely that the achievements of one
field will inform the others, and it must be the case that any
discrepancies between fields can be ultimately resolved. Given
the necessarily interdisciplinary nature of this discussion, the Annual
Meeting of the Cognitive Science Society is an ideal venue for these conversations.

\section{Goals and Scope}

This workshop brings together leading researchers in the language and
cognition of concepts and events -- both as speakers and as audience
members -- in order to share knowledge, discuss open research
questions of mutual concern, and shape the path forward. Precisely
because questions about the representation of concepts and events are
so broadly applicable across the cognitive sciences, they tend to be
studied somewhat independently in different (sub)disciplines. Thus,
gatherings like this one are crucial for ensuring efficient
dissemination of ideas and findings.

The workshop is organized around language acquisition, particularly
the acquisition of verbs. This will help focus discussion without
necessarily sacrificing breadth: Verb acquisition presents a
particularly rich set of phenomena touching upon issues of central
concern to the disparate concepts and events literatures. To these ends, the workshop speakers represent diverse research
traditions and topics. Many have contributed to multiple of these literatures.

\jtodo{I don't love the following paragraph, but after 90
  min. struggling with it, this is the best I got!}

Thus, Cynthia Fisher, Joshua Hartshorne, and
Melissa Kline in particularly will discuss their work on grounding the acquisition of
verbs in event representations
\cite<cf.>{Connor2013,Fisher2010,Klineinpress,HOSULS}, and Noah Goodman
will discuss his work grounding language processing more generally in
conceptual representations
\cite<cf.>{Piantadosiinpress,Kao2014}. Beyond this, many of the talks will focus
squarely on conceptual and event representations. Barbara Tversky and Jeffrey
Zacks will discuss event perception and segmentation \cite<cf.>{Radvansky2014,Tversky2013}. Joshua Hartshorne,
Beth Levin, and Eva Wittenberg will discuss insights into event
representations that stem from investigation of linguistic structure
\cite<cf.>{Wittengerg2014,HOSULS,Levin2005}. The role of intuitive
theories in concepts and thought will be discussed by Noah Goodman and
Joshua Tenenbaum \cite<cf.>{Battaglia2013,Goodman2011,Ettinger,HOT2015}. Many speakers
will discuss recent work on infant and child cognition, especially Dare Baldwin and
Gergely Csibra \cite<cf.>{Henrik2015,Badwin2013}.

All speakers will endeavor to draw out connections between the
different lines of research. Question periods and discussion during
coffee breaks will allow participants -- including audience members --
to discuss convergences and apparent conflicts between the different
literatures, generating future directions for research.

\section{Workshop Organization}

The workshop will be organized around a set of thirty-minute
presentations (including Q\&A). The presentations will range from
theoretical overviews to detailed discussion of specific
phenomena. Interspersed coffee breaks will help spur discussion about
promising avenues for future research and help build a common
vocabulary and agenda.

\subsection{Workshop Organizers}

\textbf{Eva Wittenberg} is \jtodo{Please blurb}. \textbf{Melissa Kline} is
\jtodo{please blurb}. \textbf{Joshua K. Hartshorne} is an assistant professor of
Psychology at Boston College. His work focuses on the interaction
between conceptual and linguistic representations in both processing
and acquisition \cite{HOSULS,HOT2015}.

\section{Target Audience}

The target audience for this workshop overlaps significantly
with the target audience of CogSci. The workshop's central themes
(language acquisition and conceptual representation) have long been
central concerns of the Society and are typically well-represented at
its meetings. Moreover, our specific focus on event representations
dovetails this year's overall conference theme: ``Recognizing and
representing events: Integrating psychological, linguistic,
computational and neural perspectives.''

Moreover, the workshop approaches these themes from a
multidisplinary perspective, as seen in the disciplinary diversity of
the participants. Because the presentations will be geared towards an
interdisciplinary audience, they should be approachable by a broad
cognitive science audience.

\section{Confirmed Speakers}
  
\href{http://baldwinlab.uoregon.edu/dr-dare-baldwin/}{Dare Baldwin}, Department of Psychology, University of Oregon
\href{https://people.ceu.edu/gergely_csibra}{Gergely Csibra}, Department of Cognitive Science, Cognitive Development Center Director, Central European University  
\href{http://www.psychology.illinois.edu/people/clfishe}{Cynthia Fisher}, Department of Psychology, University of Illinois, Urbana-Champaign
  
  Allison Gopnik XYZ not confirmed yet, included for spacing reasons
  
\href{http://cocolab.stanford.edu/ndg.html}{Noah Goodman}, Department of Psychology, Stanford University
\href{http://joshuakhartshorne.org/}{Joshua Hartshorne}, Department of Psychology, Boston College
\href{http://www.melissakline.net/}{Melissa Kline}, Department of Psychology, Harvard University\href{http://web.stanford.edu/~bclevin/}{Beth Levin}, Department of Linguistics, Stanford University*
\href{http://web.mit.edu/cocosci/josh.html}{Josh Tenenbaum}, Department of Brain and Cognitive Sciences, Massachusetts Institute of Technology*
\href{http://www-psych.stanford.edu/~bt/}{Barbara Tversky}, Department of Psychology, Stanford University
\href{http://evawittenberg.com/i/start.html}{Eva Wittenberg}, Departments of Linguistics and Psychology, University of California, San Diego  
\href{http://pages.wustl.edu/dcl/jeff-zacks/}{Jeff Zacks}, Department of Psychological and Brain Sciences, Washington University in St. Louis


\bibliographystyle{apacite}

\setlength{\bibleftmargin}{.125in}
\setlength{\bibindent}{-\bibleftmargin}

\bibliography{CogSci_Template}


\end{document}

%%% Local Variables:
%%% mode: latex
%%% TeX-master: t
%%% End:
