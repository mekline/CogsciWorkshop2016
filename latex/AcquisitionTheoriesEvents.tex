%
% Annual Cognitive Science Conference
% Sample LaTeX Two-Page Summary -- Proceedings Format
%

% Original : Ashwin Ram (ashwin@cc.gatech.edu)       04/01/1994
% Modified : Johanna Moore (jmoore@cs.pitt.edu)      03/17/1995
% Modified : David Noelle (noelle@ucsd.edu)          03/15/1996
% Modified : Pat Langley (langley@cs.stanford.edu)   01/26/1997
% Latex2e corrections by Ramin Charles Nakisa        01/28/1997
% Modified : Tina Eliassi-Rad (eliassi@cs.wisc.edu)  01/31/1998
% Modified : Trisha Yannuzzi (trisha@ircs.upenn.edu) 12/28/1999 (in process)
% Modified : Mary Ellen Foster (M.E.Foster@ed.ac.uk) 12/11/2000
% Modified : Ken Forbus                              01/23/2004
% Modified : Eli M. Silk (esilk@pitt.edu)            05/24/2005
% Modified : Niels Taatgen (taatgen@cmu.edu)         10/24/2006
% Modified : David Noelle (dnoelle@ucmerced.edu)     11/19/2014

%% Change "letterpaper" in the following line to "a4paper" if you must.

\documentclass[10pt,letterpaper]{article}

\usepackage{cogsci}
\usepackage{pslatex}
\usepackage{apacite}
\usepackage{color}


% \usepackage[disable]{todonotes} % notes not showed
\usepackage[]{todonotes}   % notes showed

\newcommand{\mtodo}[2][]
{\todo[caption={#2}, size=\small, #1, color = green, inline]{\renewcommand{\baselinestretch}{1}\selectfont \textbf{TG}: #2}~}

\newcommand{\jtodo}[2][]
{\todo[caption={#2}, size=\small, #1, color = orange, inline]{\renewcommand{\baselinestretch}{1}\selectfont \textbf{JH}: #2}~}

\newcommand{\etodo}[2][]
{\todo[caption={#2}, size=\small, #1, color = red, inline]{\renewcommand{\baselinestretch}{1}\selectfont \textbf{LN}: #2}~}

\title{Learning to Talk about Events\\ Grounding Language Acquisition in Intuitive Theories and Event Cognition} %Josh: "Talking about thoughts" for title pt. 1c

\author{{\large \bf Eva Wittenberg (ewittenberg@ucsd.edu)} \\
  Department of Linguistics,
  University of California, San Diego \\
  9500 Gilman Dr.,
  La Jolla, CA 92093-0108 USA \\
  \AND {\large \bf Melissa Kline (mekline@mit.edu)} \\
  Department of Psychology,
  Harvard University \\
  33 Kirkland St.,
  Cambridge, MA 02138 USA \\
  \AND {\large \bf Joshua K. Hartshorne (joshua.hartshorne@bc.edu)} \\
  Department of Psychology,
  Boston College \\
  140 Commonwealth Ave,
  Chestnut Hill, MA 02467}

\usepackage{navigator} %use instead of hyperref, which has lots of incompatibilities

\begin{document}

\maketitle

\begin{quote}
\small
\textbf{Keywords:}
language acquisition; concepts; event cognition; cognitive
development; intuitive theories; argument structure
\end{quote}
%mk removed 'language' for space, are we gonna miss anyone who won't otherwise find the workshop?

\section{Introduction}

Language is a powerful tool for moving thoughts from the
head of one person to another. Thus, any theory of language must make
contact with theories of thought and conceptual representation. Any
theory of language acquisition must explain how children link words to
concepts, whether those concepts were pre-existing or created during
the process of language acquisition. Conversely, theories of
conceptual representation are constrained by the need to support
language and language acquisition. Indeed, there is a rich tradition
of cognitive science research that explicitly tackles these issues \cite{Clark2004,Bowerman1989}.

This is an opportune time to assess our
current understanding of how language acquisition might be grounded in
thought -- particularly focusing beyond the better-explored domains of
objects and kinds to events. Recent years have seen the emergence of a
robust psychological literature on event representations \cite{Tversky2013}. There has
likewise been explosive growth in
``intuitive theory'' or ``Theory Theory'' approaches to conceptual representation
\cite{Gopnikinpress,HOT2015,Goodman2011,Battaglia2013}. On such accounts, concepts are
embedded in robust theories -- much like those used by
scientists -- and derive their meaning from their roles in those
theories. Importantly, our growing understanding of concepts
is not limited to mature representations: Work on infant and child
conceptual representations has been particularly productive \cite{Gopnikinpress,Ettinger,Henrik2015}.

Language research has also seen a flourishing of theories of
concepts. Linguists now have well-specified, articulated theories of
the semantics of verbs, drawing on representations of abstract
concepts such as events \cite{Levin2011,Levin2005}. Importantly, there is now increasingly strong evidence for their psychological reality and role in language acquisition
\cite{Ambridge2013,HOSULS}. While these theories bear certain
similarities to the infant cognition work (e.g., representations of
agency, intentionality, and causation play key roles), in other ways
they diverge (e.g., they have no clear analog to an intuitive theory).

This workshop brings linguists and language acquisition researchers
together with experts in event cognition, intuitive theories, and
cognitive development in order to understand how
these independent but linked fields inform one
another. Because language makes direct contact with thought (and vice versa) it is likely that the achievements of one
field will inform the others, and discrepancies between fields must
ultimately be resolved in order to adequately explain phenomena in both areas. Given
the interdisciplinary nature of this discussion, the Annual
Meeting of the Cognitive Science Society is an ideal venue.

\section{Goals and Scope}

This workshop brings together leading researchers in the language and
cognition of concepts and events -- both as speakers and as audience
members -- in order to share knowledge, discuss open research
questions of mutual concern, and shape the path forward. Precisely
because questions about the representation of concepts and events are
so broadly applicable across the cognitive sciences, they tend to be
studied somewhat independently in different (sub)disciplines. Thus,
gatherings like this one are crucial for ensuring efficient
dissemination of ideas and findings.

The workshop is organized around language acquisition, particularly
the acquisition of verbs. This will help focus discussion without
necessarily sacrificing breadth: Verb acquisition presents a
particularly rich set of phenomena touching upon issues of central
concern to the disparate concepts and events literatures. To these ends, the workshop speakers represent diverse research
traditions and topics, and many have contributed to multiple of these literatures.

Beth Levin, Joshua Hartshorne, and Eva Wittenberg will discuss insights into event
representations that stem from investigation of \textbf{linguistic structure}. Cynthia Fisher, Joshua Hartshorne, and
Melissa Kline will discuss the grounding of \textbf{verb acquisition} in event representations, and Noah Goodman
will discuss his work grounding language processing more generally in
conceptual representations. Beyond this, many of the talks will focus squarely on {conceptual and event representations}. Many speakers
will discuss recent work on \textbf{infant and child cognition}, especially Dare Baldwin and
Gergely Csibra. Barbara Tversky and Jeffrey Zacks will discuss event
perception and segmentation, and the potential role of
\textbf{intuitive theories} will be discussed by Noah Goodman and
Joshua Tenenbaum.

All speakers will endeavor to draw out connections between the
different lines of research. Question periods and discussion during
coffee breaks will allow participants and attendees
to synthesize the different literatures, generating future directions
for research.

\section{Workshop Organization}

The workshop will be organized around a set of thirty-minute
presentations (including Q\&A). The presentations will range from
theoretical overviews to detailed discussion of specific
phenomena. Interspersed coffee breaks will help spur discussion about
promising avenues for future research and help build a common
vocabulary and agenda. Presentations will be geared towards an
interdisciplinary audience and should be approachable by a broad
cognitive science audience.

\subsection{Workshop Organizers}

\textbf{Eva Wittenberg} is a postdoctoral researcher in the
Departments of Linguistics and Psychology at UCSD. Her research
focuses on how the mind assembles meaning and how that capacity came
to be. \textbf{Melissa Kline} is a postdoctoral researcher in the Harvard Psychology department studying how babies' and young children's cognitive representations of events relate to verb meaning and argument structure.
\textbf{Joshua K. Hartshorne} is an assistant professor of
Psychology at Boston College. His work focuses on the interaction
between conceptual and linguistic representations.

\section{Target Audience}

The target audience for this workshop overlaps significantly
with the target audience of CogSci. The workshop's central themes
(language acquisition and conceptual representation) are central concerns of the Society and are typically well-represented at
its meetings. The multidisciplinary nature of the work is particularly
appropriate for a multidisciplinary conference like CogSci. Finally,
our specific focus dovetails this year's overall conference theme: ``Recognizing and
representing events: Integrating psychological, linguistic,
computational and neural perspectives.''

\section{Confirmed Speakers}

{\color{blue} \urllink{http://baldwinlab.uoregon.edu/dr-dare-baldwin/}{Dare Baldwin}}, University of Oregon\\
{\color{blue} \urllink{https://people.ceu.edu/gergely_csibra}{Gergely Csibra}}, Central European University\\
{\color{blue} \urllink{http://www.psychology.illinois.edu/people/clfishe}{Cynthia Fisher}}, University of Illinois, Urbana-Champaign\\
{\color{blue} \urllink{http://cocolab.stanford.edu/ndg.html}{Noah Goodman}}, Stanford University\\
{\color{blue} \urllink{http://joshuakhartshorne.org/}{Joshua Hartshorne}}, Boston College\\
{\color{blue} \urllink{http://www.melissakline.net/}{Melissa Kline}}, Harvard University\\
{\color{blue} \urllink{http://web.stanford.edu/~bclevin/}{Beth Levin}}, Stanford University\\
{\color{blue} \urllink{http://web.mit.edu/cocosci/josh.html}{Josh Tenenbaum}}, Massachusetts Institute of Technology\\
{\color{blue} \urllink{http://www-psych.stanford.edu/~bt/}{Barbara Tversky}}, Columbia University\\
{\color{blue} \urllink{http://evawittenberg.com/i/start.html}{Eva Wittenberg}}, University of California, San Diego\\
{\color{blue} \urllink{http://pages.wustl.edu/dcl/jeff-zacks/}{Jeff Zacks}}, Washington University in St. Louis\\
%Version of the speaker list with departments listed, but I don't think it's necessary
%\begin{itemize}
%\item[] \urllink{http://baldwinlab.uoregon.edu/dr-dare-baldwin/}{Dare Baldwin}, Department of Psychology, University of Oregon
%\item[] \urllink{https://people.ceu.edu/gergely_csibra}{Gergely Csibra}, Department of Cognitive Science, Central European University
%\item[] \urllink{http://www.psychology.illinois.edu/people/clfishe}{Cynthia Fisher}, Department of Psychology, University of Illinois, Urbana-Champaign
%
%\item[] Allison Gopnik XYZ not confirmed yet, included for spacing reasons
%
%\item[] \urllink{http://cocolab.stanford.edu/ndg.html}{Noah Goodman}, Department of Psychology, Stanford University
%\item[] \urllink{http://joshuakhartshorne.org/}{Joshua Hartshorne}, Department of Psychology, Boston College
%\item[] \urllink{http://www.melissakline.net/}{Melissa Kline}, Department of Psychology, Harvard University\item[] \urllink{http://web.stanford.edu/~bclevin/}{Beth Levin}, Department of Linguistics, Stanford University*
%\item[] \urllink{http://web.mit.edu/cocosci/josh.html}{Josh Tenenbaum}, Department of Brain and Cognitive Sciences, Massachusetts Institute of Technology*
%\item[] \urllink{http://www-psych.stanford.edu/~bt/}{Barbara Tversky}, Department of Psychology, Stanford University
%\item[] \urllink{http://evawittenberg.com/i/start.html}{Eva Wittenberg}, Departments of Linguistics and Psychology, University of California, San Diego
%\item[] \urllink{http://pages.wustl.edu/dcl/jeff-zacks/}{Jeff Zacks}, Department of Psychological and Brain Sciences, Washington University in St. Louis
%\end{itemize}
\bibliographystyle{apacite}

\setlength{\bibleftmargin}{.125in}
\setlength{\bibindent}{-\bibleftmargin}

\bibliography{CogSci_Template}


\end{document}
\finishpdffile
%%% Local Variables:
%%% mode: latex
%%% TeX-master: t
%%% End:
