%
% Annual Cognitive Science Conference
% Sample LaTeX Two-Page Summary -- Proceedings Format
%

% Original : Ashwin Ram (ashwin@cc.gatech.edu)       04/01/1994
% Modified : Johanna Moore (jmoore@cs.pitt.edu)      03/17/1995
% Modified : David Noelle (noelle@ucsd.edu)          03/15/1996
% Modified : Pat Langley (langley@cs.stanford.edu)   01/26/1997
% Latex2e corrections by Ramin Charles Nakisa        01/28/1997
% Modified : Tina Eliassi-Rad (eliassi@cs.wisc.edu)  01/31/1998
% Modified : Trisha Yannuzzi (trisha@ircs.upenn.edu) 12/28/1999 (in process)
% Modified : Mary Ellen Foster (M.E.Foster@ed.ac.uk) 12/11/2000
% Modified : Ken Forbus                              01/23/2004
% Modified : Eli M. Silk (esilk@pitt.edu)            05/24/2005
% Modified : Niels Taatgen (taatgen@cmu.edu)         10/24/2006
% Modified : David Noelle (dnoelle@ucmerced.edu)     11/19/2014

%% Change "letterpaper" in the following line to "a4paper" if you must.

\documentclass[10pt,letterpaper]{article}

\usepackage{cogsci}
\usepackage{pslatex}
\usepackage{apacite}

% \usepackage[disable]{todonotes} % notes not showed
\usepackage[]{todonotes}   % notes showed

\newcommand{\mtodo}[2][]
{\todo[caption={#2}, size=\small, #1, color = green, inline]{\renewcommand{\baselinestretch}{1}\selectfont \textbf{TG}: #2}~}

\newcommand{\jtodo}[2][]
{\todo[caption={#2}, size=\small, #1, color = orange, inline]{\renewcommand{\baselinestretch}{1}\selectfont \textbf{JH}: #2}~}

\newcommand{\etodo}[2][]
{\todo[caption={#2}, size=\small, #1, color = red, inline]{\renewcommand{\baselinestretch}{1}\selectfont \textbf{LN}: #2}~}

\title{Grounding Language Acquisition in Intuitive Theories and Event Cognition}

\author{{\large \bf Eva Wittenberg (ewittenberg@ucsd.edu)} \\
  Department of Linguistics,
  University of California, San Diego \\
  9500 Gilman Dr.,
  La Jolla, CA 92093-0108 USA \\
  \AND {\large \bf Melissa Kline (mekline@mit.edu)} \\
  Department of Psychology,
  Harvard University \\
  33 Kirkland St.,
  Cambridge, MA 02138 USA \\
  \AND {\large \bf Joshua K. Hartshorne (joshua.hartshorne@bc.edu)} \\
  Department of Psychology,
  Boston College \\
  140 Commonwealth Ave,
  Chestnut Hill, MA 02467}


\begin{document}

\maketitle

\begin{quote}
\small
\textbf{Keywords:}
language; language acquisition; concepts; event cognition; cognitive
development; intuitive theories; argument structure
\end{quote}

\section{Introduction}



FUBAR

\section{Goals and Scope}

FUBAR

\section{Workshop Organization}

\noindent The workshop will be organized around a set of thirty-minute
presentations (including Q\&A). The presentations will range from
theoretical overviews to detailed discussion of specific
phenomena. Interspersed coffee breaks will help spur discussion about
promising avenues for future research and help build a common
vocabulary and agenda.

\subsection{Workshop Organizers}

\textbf{Eva Wittenberg} is \jtodo{Please blurb}. \textbf{Melissa Kline} is
\jtodo{please blurb}. \textbf{Joshua K. Hartshorne} is an assistant professor of
Psychology at Boston College. His work focuses on the interaction
between conceptual and linguistic representations in both processing
and acquisition \cite{HOSULS,HOT2015}.

\section{Target Audience}

\noindent The target audience for this workshop overlaps significantly
with the target audience of CogSci. The workshop's central themes
(language acquisition and conceptual representation) have long been
central concerns of the Society and are typically well-represented at
its meetings. Moreover, our specific focus on event representations
dovetails this year's overall conference theme: ``Recognizing and
representing events: Integrating psychological, linguistic,
computational and neural perspectives.''

\noindent Moreover, the workshop approaches these themes from a
multidisplinary perspective, as seen in the disciplinary diversity of
the participants. Because the presentations will be geared towards an
interdisciplinary audience, they should be approachable by a broad
cognitive science audience.

\section{Acknowledgments}

\jtodo{Melissa and Eva: List funding. I don't think we have anyone to
  thank. My funding right now is start-up, which isn't usually acknowledged.}

\bibliographystyle{apacite}

\setlength{\bibleftmargin}{.125in}
\setlength{\bibindent}{-\bibleftmargin}

\bibliography{CogSci_Template}


\end{document}

%%% Local Variables:
%%% mode: latex
%%% TeX-master: t
%%% End:
